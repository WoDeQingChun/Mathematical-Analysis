\chapter{常用不等式}
\section{Bernoulli不等式}
\begin{proposition}[Bernoulli不等式]
    设$h>-1$,$ n \in N^+ $, 则成立 $ (1+h)^n \geq 1 + nh $, 且当 $ n > 1 $时当且仅当 $ h = 0 $时取等号
\end{proposition}
\begin{proof}
    提供两种证明思路:
    \begin{enumerate}
        \item 数学归纳法容易证明,在此不详细叙述
        \item 注意到有公式
        \begin{equation*}
            a^n - b^n = (a-b)(a^{n-1} + a^{n-2}b + \cdots + b^{n-1})
        \end{equation*}
        成立.对原不等式作变形、因式分解,再对 $ h $分类讨论容易证明
    \end{enumerate}
\end{proof}

\begin{proposition}[推广]
    设 $ a_i > -1(i = 1, \cdots, n) $且同号,则成立
    \begin{equation*}
        \prod_{i=1}^{n}(1+a_i) \geq 1 + \sum_{i=1}^{n}a_i
    \end{equation*}
    当且仅当 $ a_1 = a_2 = \cdots = a_n = 0 $时取等号
\end{proposition}
\begin{proof}
    提供两种证明思路:
    \begin{enumerate}
        \item 数学归纳法是容易证明的
        \item 直接对原不等式左边进行展开有:
        \begin{equation*}
            \begin{split}
                \prod_{i=1}^{n}(1+a_i) 
                &= 1 + \sum_{i=1}^{n}a_i + \sum_{i_1 i_2}^{}a_{i_1 i_2} + \cdots + \sum_{i_1 \cdots i_n}^{}a_{i_1} \cdots a_{i_n}   \\
                &= 1 + \sum_{i=1}^{n}a_i + \sum_{k=2}^{n}\sum_{i_1 \cdots i_k}^{}a_{i_1} \cdots a_{i_k}\text{(其中$i_1 \cdots i_k$是$1,2,\cdots,n$的一个$n$元有序排列)}\\
                &\geq \text{右边}
            \end{split}
        \end{equation*}
    \end{enumerate}
\end{proof}

为了方便应用,引出Bernoulli不等式的\emph{双参数形式}
\begin{proposition}[Bernoulli不等式的\emph{双参数形式}]
    设 $ A > 0 $, $ A + B > 0 $,则 $ 1 + \frac{B}{A} > 0$, 应用Bernolli不等式有:
    \begin{equation*}
        \left( 1 + \frac{B}{A} \right)^n \geq 1 + n \cdot \frac{B}{A}
    \end{equation*}
    化简得双参数形式的Bernoulli不等式:
    \begin{equation*}
        (A + B)^n \geq A^n + nA^{n-1}B
    \end{equation*}
    易知当且仅当$\frac{B}{A} = 0$即$B = 0$时取等号
\end{proposition}

\section{Cauchy-Schwarz不等式与三角不等式}
\subsection*{Cauchy-Schwarz不等式}
\begin{proposition}
    Cauchy-Schwarz不等式有多种形式,在此先弄懂在欧氏空间$R^n$中的形式。它的几何含义是:$\forall a, b \in R^n$,都有$|a \cdot b| \leq |a||b|$.
    若设$a = (a_1, \cdots, a_n)$, $b = (b_1, \cdots, b_n)$,则有:
    \begin{equation*}
        \left| \sum_{i=1}^{n}a_ib_i \right| \leq \sqrt{\sum_{i=1}^{n}a_{i}^{2}} \sqrt{\sum_{i=1}^{n}b_{i}^{2}}
    \end{equation*}
\end{proposition}

这个不等式的证明有点神来之笔的意思,暂且看懂就好
\begin{proof}
    引入变量$\lambda$,注意到:
    \begin{equation*}
        0 \leq \sum_{i=1}^{n}(\lambda a_i - b_i)^2  = \lambda^2 \sum_{i=1}^{n}a_i^2 - 2\lambda \sum_{i=1}^{n}a_i b_i 
        + \sum_{i=1}^{n}b_i^2
    \end{equation*}
    这是一个关于变量$\lambda$的一元二次不等式,根据其根的判别式可直接得到待证不等式。
\end{proof}

\subsection*{三角不等式}
这个不等式不仅在实数范围内成立,在复数范围和欧式空间中也成立。在欧式空间中两个不等号的含义分别是:
\begin{itemize}
    \item 三角形两边之差小于第三边
    \item 三角形两边之和大于第三边
\end{itemize}
\begin{equation*}
    | |a|-|b| | \leq |a-b| \leq |a| + |b|
\end{equation*}
在复数范围和欧式空间中的证明都只要借助Cauchy-Schwarz不等式即可

\section{均值不等式}
设$a_i > 0\quad (i = 0, 1, \cdots, n)$,则有算数平均值不小于几何平均值不小于调和平均值,即:
\begin{equation*}
    \frac{1}{n}\sum_{i=1}^{n}a_i \geq \sqrt[n]{ \prod_{i=1}^{n}a_i } \geq \frac{1}{\frac{1}{n}\sum_{i=1}^{n}\frac{1}{a_1}}
\end{equation*}
证明思路如下:
\begin{itemize}
    \item 第二个不等式可由第一个不等式证明
    \item 第一个不等式由Cauchy向前向后证明法证明
\end{itemize}

\section{排序不等式}
设有两个有序实数组:$ a_1 \leq \cdots \leq a_n $, $ b_1 \leq \cdots \leq b_n $, $ c_1, \cdots,  c_n $是
$ b_1, \cdots,  b_n $的任一排列,则有
\begin{equation*}
    \text{顺序和} \geq \text{乱序和} \geq \text{逆序和}
\end{equation*}
其中:
\begin{itemize}
    \item 顺序和:$ \sum_{i=1}^{n}a_i b_i $
    \item 乱序和:$ \sum_{i=1}^{n}a_i c_i $
    \item 逆序和:$ \sum_{i=1}^{n}a_i b_{n+1-i} $
\end{itemize}
\begin{remark}
    这是一种贪心的算法思想:要使结果最大,每一步都选出两个序列中最大的数相乘
\end{remark}
\begin{proof}
    TODO:结论显然,证明以后有时间再添加
\end{proof}

\section{重要的三角不等式}

这个不等式是重要极限的基础
\begin{proposition}
    若 $ 0 < x < \frac{\pi}{2} $, 则 $ sin\,x < x < tan \, x $
\end{proposition}
\begin{proof}
    证明要点:
    \begin{itemize}
        \item 直角三角形的直角边小于斜边
        \item 单位圆面积法
    \end{itemize}
\end{proof}
\section{$n!$的估计}

$n!$的常见估计很有用,在此先介绍几个简单的估计方式:

\begin{proposition}
    $n > 1$时,有:
    \begin{equation*}
        n! < \left( \frac{n+1}{2} \right)^n    
    \end{equation*}
    这由算数-几何均值不等式可立即得到
\end{proposition}

更精确的上界的估计:
注意到:
\begin{equation*}
    (n!)^2 = (n \cdot 1)((n-1)\cdot 2) \cdots (1 \cdot n)
\end{equation*}
对上述等式右边的$n$项应用算数-几何均值不等式容易得到:
\begin{equation*}
    n! < \left( \frac{n+2}{\sqrt{6}} \right)^n
\end{equation*}
后面还会介绍Wallis公式和Stirling公式来估计$n!$