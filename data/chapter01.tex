\chapter{实数的连续性}
整个微积分的基础建立在实数的连续性之上。但不幸的是实数的连续性是一个很显然却又比较复杂的事情(涉及到实数系的构造)。
而确界存在定理又很好地反映了实数不存在空隙。为了方便直接从确界存在定理(\emph{默认对,不加证明})出发进行学习。

\begin{theorem}[确界存在定理]
    \begin{itemize}
        \item 实数的非空子集若有上界,必有最小上界(上确界)
        \item 实数的非空字节若有下界,必有最大下界(下确界)
    \end{itemize}
\end{theorem}